\section{基于类的组织}

基于类的组织源于这样的观察:分派矩阵对每个作用在类上的方法可以被“析出”实例数据以获得\textit{类向量}(\textit{class vector})$ e_{\mathtt{cv}} $,它的类型如下:

$$ \tau_{\mathtt{cv}} \triangleq \prod_{c \in C}(\tau^{c} \rightharpoonup (\prod_{d \in D} \rho_{d})) $$

每个类向量包含一个\textit{构造器}(\textit{constructor}),它能确定每个作用在给定实例数据的方法的结果类型。

一个类型为 $ \rho = \prod_{d \in D} \rho_{d} $ 的对象包含方法的结果类型之积。比如,在平面坐标点的例子里,$ \rho $ 具有积类型:

$$ \langle \mathtt{dist} \hookrightarrow \rho_{\mathtt{dist}}, \mathtt{quad} \hookrightarrow \rho_{\mathtt{quad}} \rangle $$

每个部分表明了作用在实例数据上的方法的结果类型。

信息的发送操作 $e \Leftarrow d $ 其实是投影 $ e \cdot d $。所以在平面坐标点的例子中,$e \Leftarrow \mathtt{dist} $ 是投影 $ e \cdot \mathtt{dist} $ ,同理 $ e \Leftarrow \mathtt{quad} $ 为投影 $ e \cdot \mathtt{quad} $。

基于类的组织将把类转化成类向量的实现,以及对于每个类 $c \in C$ 包含具有类型 $\tau^{c} \hookrightarrow \rho $ 的构造器的类型元组 $ \tau_{\mathtt{cv}} $ 相结合。
类向量由 $ e_{\mathtt{cv}} = \langle c \rightharpoonup e^{c} \rangle_{c \in C} $ 定义,其中对于每个 $c \in C$ 表达式 $ e^{c} $ 为:

$$ \lambda(u : \tau^{c})\langle d \hookrightarrow e_{\mathtt{dm}} \cdot c \cdot d(u) \rangle_{d \in D} $$

比如,对于类 $ \mathtt{cart} $ 的构造器函数 $e^{\mathtt{cart}}$ 形式为:

$$ \lambda(u : \tau^{\mathtt{cart}}) \langle \mathtt{dist} \hookrightarrow e_{\mathtt{dm}} \cdot \mathtt{cart} \cdot \mathtt{dist}(u), \mathtt{quad} \hookrightarrow e_{\mathtt{dm}} \cdot \mathtt{cart} \cdot \mathtt{quad}(u) \rangle $$

同理,类 $ \mathtt{pol} $ 的构造器函数 $e^{\mathtt{pol}}$ 为:

$$ \lambda(u : \tau^{\mathtt{pol}}) \langle \mathtt{dist} \hookrightarrow e_{\mathtt{dm}} \cdot {\mathtt{pol}} \cdot \mathtt{dist}(u), \mathtt{quad} \hookrightarrow e_{\mathtt{dm}} \cdot {\mathtt{pol}} \cdot \mathtt{quad}(u) \rangle $$

这样,类向量 $e_{\mathtt{cv}}$ 为具有类型 $ \langle {\mathtt{cart}} \hookrightarrow \tau^{\mathtt{cart}} \rightharpoonup \rho, {\mathtt{pol}} \hookrightarrow \tau^{\mathtt{pol}} \rightharpoonup \rho \rangle$

通过类的构造器作用在实例数据上,得到类的对象:

$$ {\mathtt{new}}[c](e) \triangleq e_{\mathtt{cv}} \cdot c(e) $$

比如,一个笛卡尔坐标系的点可以由 $ {\mathtt{new}}[{\mathtt{cart}}](\langle {\mathtt{x}} \hookrightarrow x_{0}, {\mathtt{y}} \hookrightarrow y_{0} \rangle) $ 获得,它的定义为:

$$ e_{\mathtt{cv}} \cdot {\mathtt{cart}}(\langle {\mathtt{x}} \hookrightarrow x_{0}, {\mathtt{y}} \hookrightarrow y_{0}\rangle) $$

同理,极坐标系的点可以由 ${\mathtt{new}}[{\mathtt{pol}}](\langle {\mathtt{r}} \hookrightarrow r_{0}, {\mathtt{th}} \hookrightarrow \theta_{0} \rangle) $ 获得,其定义为:

$$ e_{\mathtt{cv}} \cdot {\mathtt{pol}}(\langle {\mathtt{r}} \hookrightarrow r_{0}, {\mathtt{th}} \hookrightarrow \theta_{0}\rangle) $$

对于点的组织的检查很容易,对于每个类 $c$ 和方法 $d$,我们可以提取出:

\begin{equation}
\begin{aligned}
 ({\mathtt{new}[c](e)}) \Leftarrow d & \longmapsto^{*} (e_{cv} \cdot c(e)) \cdot d \\ 
 & \longmapsto^{*} e_{dm} \cdot c \cdot d(e)
 \nonumber
 \end{aligned}
 \end{equation}

也就是说,信息发送会唤起给定对象上实例数据上给定方法的行为。
