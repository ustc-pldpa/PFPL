通常,一个类型的各个值会被划分若干\textit{类} (\textit{class})中,每一类的数据分别具有不同的内在结构。以平面上的坐标点为例,可以分为笛卡尔坐标系形式的类与极坐标系形式的类,它们都由一对实数表示。在笛卡尔坐标系中,这两的数字对应着点的 $ x $ 与 $ y $ 的坐标;但在极坐标系中,它们对应着点的级径 $ r $ 以及与极角 $ \theta $。每个被分类的值叫做该类的一个 \textit{对象}(\textit{object})或 \textit{实例} (\textit{instance})。类决定了被分类的数据的类型,即该类的\textit{实例类型}(\textit{instance type});该数据本身称之为对象的\textit{实例数据}(\textit{instance data})。

\textit{方法}(\textit{method})是作用在被分类数据上的函数。方法的行为由它参数的类决定\footnote[1]{简单起见,我们假定可以同时分派多个参数的类。我们只关注单一的分派。}。方法的\textit{分派}(\textit{dispatch})由参数的类别决定。由于这种选择是运行时发生的,因此我们称其为\textit{动态分派}(\textit{dynamic dispatch})。比如,用来计算点到原点距离平方的方法在两种坐标系下表现不同。在笛卡尔坐标系中,距离的平方为 $x^2 + y^2$,而在极坐标系中为 $r^2$;再比如,判断点的所在象限时,笛卡尔坐标系的点需要考察 $ x $ 与 $ y $ 的符号,而在极坐标系中需要考察 $ \theta $ 除以 $ \pi / 2 $。

动态分派经常用一类特定的实现策略描述,我们称之为\textit{基于类的}(\textit{class-based})组织。在这种组织中,每个对象被描述为作用在该对象的上的一列方法。有时我们也用一种与其等价的组织,\textit{基于方法的}(\textit{method-based})组织,其中方法的行为随着作用对象的类型不同而不同。不管是哪一种组织,核心想法都是:(1)对象总是属于某一类,(2)方法在类上进行分派。基于类的与基于方法的组织之间是可以相互转化的,实际上,它们是加类型和积类型的天然对偶关系。通过观察每个对象上的每个方法的行为,我们可以解释这种对称性,并称其为\textit{分派矩阵}(\textit{dispatch matrix})。由分派矩阵,我们会发现这两种组织的对称关系是显然的。
