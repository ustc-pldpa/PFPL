\chapter{抽象类型}

数据抽象可能是构建程序最重要的技术。
其主要思想是引入一个\textit{接口},作为\textit{客户端}和抽象类型\textit{实现器}之间的协议。
接口指明了客户端依赖什么和实现器必须提供什么来满足协议。
该接口用于隔离客户端与实现器,以便客户端和实现器能够独立开发。
特别是如果两个具有相同接口的实现相互模拟关于接口的操作,那么一个实现可以被另一个实现替代而不影响客户端的行为。
该属性称为抽象类型的\textit{表示独立性}。

数据抽象是用\textit{存在类型}扩展的\textbf{F}语言来形式化的。
接口是存在类型,它提供了一个对未指定的或抽象的类型进行操作的集合。
实现是存在类型的引入形式——包,而客户端是相应消去形式的使用者。
值得注意的是,数据抽象的编程概念自然而然地被存在类型量化的逻辑概念所捕获。
存在类型与普遍的类型都密切相关,因此经常一起处理。
表面的原因是两者都是类型量化的形式,因此都需要类型变量的机制。
更深层次的原因是存在类型可以从普遍类型定义——令人惊讶的是,数据抽象实际上只是多态的一种形式!
因此,表示独立性是第16章讨论的多态函数的参数性质的一个应用。


\subimport{./}{existential-types}
\subimport{./}{data-abstraction}
\subimport{./}{definability-of-existential-types}
\subimport{./}{representation-independence}
\subimport{./}{notes}
\subimport{./}{exercises}