\chapter{类型安全}%章节标题应该放到括号里,第几章应为\chapter{}
Chapter 6

类型安全

绝大部分编程语言是安全的(类型安全,或强类型的。非正式地说,这意味着执行过程)%少了右括号

中不会出现某些不匹配。举例说明:规定$\mathrm{E}$类型安全,绝对不会出现数字加上字符串,或连

接两个数字的情况。这两个操作都是没有意义的。%都是没有意义的

类型安全一般表述静态语义和动态语义的一致性。静态语义可以看成预计一个表达式的值具有某个形%静态和动态改为静态语义和动态语义

式,以使动态语义下的这个表达式是良定义的。因此,求值不能“卡在”在一个无可能转换的状态%evaluation应为求值,下同

下,在实现上对应执行时“非法指令”错误的缺失。通过显示每一步保留可类型化的转换,或

显示可类型化的规定是良定义的,安全性可以得到证明。因此,求值绝不会“消失”,也绝不

会遇到非法指令。

对于语言$\mathrm{E}$的类型安全规定如下:

定理 6.1 (类型安全).%应使用定理环境,参考note部分,下同

{\it 1. 如果} $e:\tau  e\mapsto e'$, {\it 那么} $e'$ : $\tau.$%“且”没有显示

{\it 2. 如果} $ e:\tau$, {\it 那么} $e\iota/\mathrm{a}/$, {\it 或存在} $e val$ {\it 满足} $e\mapsto e'.$%前面应该是e val

第一条称为{\it 保留性},指求值的每一步能保持类型的归类;第二条称作{\it 进展性},保证表达式

要么已经是值,要么可以进一步求值。安全性是以上两者的结合。

我们认为表达式$e$是{\it 中止的} 当且仅当e不是值,且没有$e'$使$e\mapsto e'$. 成立。根据安全性定理,中%iff为当且仅当

止的状态一定是弱类型的。换种说法,良类型的状态不会中止。

\subimport{./}{Preservation}
\subimport{./}{Progress}
\subimport{./}{Run-Time_Errors}
\subimport{./}{notes}
\subimport{./}{exercises}











