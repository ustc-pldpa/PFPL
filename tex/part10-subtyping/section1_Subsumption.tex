\section{包容}
一个\textit{子类型断言}有着如下形式:
$\tau{'} <: \tau$
它表明$\tau{'}$是$\tau$的子类型。对于一个可采纳的子类型定型,我们至少要求如下的\textit{结构规则}成立:
%这里的admissable一词,我在纠结翻译为“可接纳”、“可采纳”、“可接受”,何者能够表达逻辑上的“隐式成立”
%这个词原本的定义是指,某个推理规则被加入形式系统后,定理集合没有发生变化
%所有合取公式即便没有这条推理规则也已经被推导出。换言之,它是多余的
%或者干脆翻译成“成立”?
\begin{subequations}
    \begin{gather}
        \frac{}{\tau<:\tau} \\
        \frac{\tau{''}<:\tau{'}\;\;\;\tau{'}<:\tau}{\tau{''}<:\tau}
    \end{gather}
\end{subequations}
事实上,我们要么不加声明地把这些规则当做原始规则,要么证明它们对于某个特定的子类型规则是可采纳的。

建立子类型关系的关键作用在于扩大类型良好的程序集合,而这一点由\textit{包容规则}完成:
\begin{equation}\label{equation:subsumption_rule}
    \frac{\Gamma\vdash\mathnormal{e}:\tau{'}\;\;\;\tau'<:\tau}{\Gamma\vdash e:\tau}
\end{equation}

与大多数定型规则相反,包容规则并\textit{不}是面向语法的,因为它并不限制$e$的形式。
换言之,包容规则可以被用诸\textit{任何}形式的表达式。
尤其是,为了论证$e:\tau$,我们有两个选择:
要么应用适合于$e$的特定形式的逻辑规则,
要么运用包孕规则检查$e:\tau{'}$和$\tau{'}<:\tau$是否都成立。


